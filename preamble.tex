%\documentclass[a4paper,12pt,oneside,pdflatex,italian,final,twocolumn]{article}
%\documentclass[a4paper,11pt,oneside,openany]{memoir} 	% Openright aabner kapitler paa hoejresider (openany = vilkaarlig/begge)


%%%% PAKKER %%%%
% ¤¤ Oversaettelse og tegnsaetning ¤¤ %
\usepackage[utf8]{inputenc}					% Input-indkodning af tegnsaet, dvs. input fra keyboard, tegnoversigt eller andet (UTF8 = Unicode)
\usepackage[T1]{fontenc}					% Output-indkodning af tegnsaet, dvs. printede fonte og tegn (T1 = Type 1 font med support for de fleste europaeiske sprog)
%\usepackage[danish]{babel}					% Sproglig fremstilling af elementer (figur vs. figure, litteratur vs. bibliography osv.)
\usepackage{csquotes}

\usepackage{booktabs}

\usepackage{changepage}

\usepackage{ragged2e,anyfontsize}			% Justering af elementer
\usepackage{lastpage}

% ¤¤ Figurer og tabeller (floats) ¤¤ %
\usepackage{graphicx} 						% Inkludering af eksterne billeder (JPG, PNG, PDF)
\usepackage{multirow}
\usepackage{multicol}% Fletning af raekker og kolonner (\multicolumn og \multirow)
\usepackage{colortbl} 						% Farver i tabeller (fx \columncolor, \rowcolor og \cellcolor)
\usepackage[dvipsnames, svgnames, table]{xcolor}        		% Definer farver med \definecolor. Se mere: http://en.wikibooks.org/wiki/LaTeX/Colors
\usepackage{flafter}						% Soerger for, at floats ikke optraeder i teksten foer deres reference
\usepackage{float}							% Muliggoer eksakt placering af floats, fx \begin{figure}[H]
\let\newfloat\relax 						% Justering mellem float-pakken og memoir
%\usepackage{eso-pic}						% Tilfoej billedekommandoer paa hver side
\usepackage{wrapfig}						% Indsaettelse af figurer omsvoebt af tekst 
%\usepackage{multicol}         	        	% Muliggoer tekst i spalter
%\usepackage{rotating}						% Rotation af tekst med \begin{sideways}...\end{sideways}
\usepackage{gensymb}
% nogle andre ting til tabeller
\renewcommand{\arraystretch}{1.5}           % Højde for hver celle
\setlength{\arrayrulewidth}{0.5mm}          % Tykkelse af streger i tabeller
\arrayrulecolor[HTML]{9B9B9B}               % Farve til streger i tabeller

% nogle ting til grafer
\usepackage{pgfplots}                       % Til at plotte grafer, Tikz baseret
    \pgfplotsset{width=10cm,compat=1.9}
\usepackage{tikz}
\usetikzlibrary{fit,backgrounds}
\usetikzlibrary{calc}
\usetikzlibrary{positioning}
\usetikzlibrary{shapes.geometric}

% ¤¤ Matematik mm. ¤¤
\usepackage{amsmath,amssymb,stmaryrd} 		% Avancerede matematik-udvidelser
\usepackage{mathtools}% Andre matematik- og tegnudvidelser
\usepackage{bm} %Matematisk fed skrift
\usepackage{mathrsfs}
\usepackage{textcomp}                 		% Symbol-udvidelser (fx promille-tegn med \textperthousand)
\usepackage{siunitx}						% Flot og konsistent praesentation af tal og enheder med \si{enhed} og \SI{tal}{enhed}
\sisetup{output-decimal-marker = {.}}		% Opsaetning af \SI og decimalseparator
\sisetup{group-separator = \ }
\sisetup{output-exponent-marker=\ensuremath{\mathrm{E}}}
\DeclareSIUnit{\flow}{flow}
\DeclareSIUnit{\frame}{frame}
\DeclareSIUnit{\degree}{deg}
\DeclareSIUnit{\baud}{Baud}


\usepackage{icomma}                         % Komma som decimalseperator fix
\usepackage{makecell}


% ¤¤ Referencer og kilder ¤¤ %
\usepackage{varioref}				% Muliggoer bl.a. krydshenvisninger med sidetal (\vref)
						% Udvidelse med naturvidenskabelige citationsmodeller, herunder Harvard-modellen


% ¤¤ Misc. ¤¤ %
\usepackage{listings}						
% Placer kildekode i dokumentet med \begin{lstlisting}...\end{lstlisting}
\usepackage{lipsum}							
% Dummy tekst med fx \lipsum[2]
\usepackage[shortlabels]{enumitem}			
% Muliggoer enkelt konfiguration af lister (se \setlist nedenfor)
\usepackage{pdfpages}						
% Goer det muligt at inkludere pdf-dokumenter med kommandoen \includepdf[pages={x-y}]{fil.pdf}	
\pdfoptionpdfminorversion=6					
% Muliggoer inkludering af pdf-dokumenter af version 1.6 og hoejere
%\pretolerance=2500 							
% Justering af afstand mellem ord (hoejt tal, mindre orddeling og mere luft mellem ord)

% Kommentarer og rettelser med \fxnote. Med 'final' i stedet for 'draft' udloeser hver note en error i den faerdige rapport.
\usepackage[footnote,draft,danish,silent,nomargin]{fixme}		

\usepackage{subcaption}

\usepackage{longtable} %makes it so tables can fill multiple pages

%%%% BRUGERDEFINEREDE INDSTILLINGER %%%%

% ¤¤ Marginer ¤¤ %
%\setlrmarginsandblock{3cm}{3cm}{*}		% \setlrmarginsandblock{Indbinding}{Kant}{Ratio}
%\setlrmarginsandblock{2.5cm}{2.5cm}{*}		%
%\setulmarginsandblock{3cm}{3.0cm}{*}		% \setulmarginsandblock{Top}{Bund}{Ratio}
%\checkandfixthelayout 						
% Oversaetter vaerdier til brug for andre pakker
%	¤¤ Afsnitsformatering ¤¤ %
\setlength{\parindent}{0mm}           		
% Stoerrelse af indryk
\setlength{\parskip}{3mm}          			
% Afstand mellem afsnit ved brug af double Enter
\linespread{1,1}							
% Linjeafstand

% ¤¤ Litteraturlisten ¤¤ %

\usepackage[
%  disable, %turn off todonotes
  colorinlistoftodos, %enable a coloured square in the list of todos
  textwidth=\marginparwidth, %set the width of the todonotes
  textsize=scriptsize, %size of the text in the todonotes
  ]{todonotes}



% ¤¤ Dybde af overskrifter ¤¤ %
%\setsecnumdepth{subsection}		 			
% Dybden af nummerede overkrifter (part/chapter/section/subsection)
% Dybden af overskrifter vist i indholdsfortegnelsen


% ¤¤ Lister ¤¤ %
\setlist{
  topsep=0pt,								
  % Vertikal afstand mellem tekst og listen
  itemsep=-1ex,								
  % Vertikal afstand mellem items
} 

% ¤¤ Visuelle referencer ¤¤ %
\usepackage[colorlinks]{hyperref}			
% Danner klikbare referencer (hyperlinks) i dokumentet
\hypersetup{colorlinks = true,				
% Opsaetning af farvede hyperlinks (interne links, citeringer og URL)
    linkcolor = black,
    citecolor = black,
    urlcolor = black
}


% ¤¤ Opsaetning af figur- og tabeltekst ¤¤ %
%\captionnamefont{\small\bfseries\itshape}	
% Opsaetning af tekstdelen ('Figur' eller 'Tabel')
%\captiontitlefont{\small}					
% Opsaetning af nummerering
%\captiondelim{. }							
% Seperator mellem nummerering og figurtekst
%\captionstyle{\centering}					
% Justering/placering af figurteksten (centreret = \centering, venstrejusteret = \raggedright)

%\captionwidth{1\linewidth}					
% Bredden af figurteksten
%\hangcaption								
% Venstrejusterer fler-linjers figurtekst under hinanden
\setlength{\belowcaptionskip}{0pt}			
% Afstand under figurteksten

% ¤¤ Opsaetning af listings ¤¤ %
%Se særskillet dokument. 
%
\definecolor{commentGreen}{RGB}{34,139,24}
\definecolor{stringPurple}{RGB}{208,76,239}
\colorlet{keyword}{blue!100!black!80}
\colorlet{comment}{green!60!black!100}



\lstset{language=Matlab,					% Sprog
	basicstyle=\ttfamily\scriptsize,		% Opsaetning af teksten
	keywords={for,if,while,else,elseif,		% Noegleord at fremhaeve
			  end,break,return,case,
			  switch,function},
	keywordstyle=\color{blue},				% Opsaetning af noegleord
	commentstyle=\color{commentGreen},		% Opsaetning af kommentarer
	stringstyle=\color{stringPurple},		% Opsaetning af strenge
	showstringspaces=false,					% Mellemrum i strenge enten vist eller blanke
	numbers=left, numberstyle=\tiny,		% Linjenumre
	extendedchars=true, 					% Tillader specielle karakterer
	columns=flexible,						% Kolonnejustering
	breaklines, breakatwhitespace=true,		% Bryd lange linjer
}

\lstdefinelanguage{VHDL}{
    %alt blåt
  morekeywords=[1]{
    library,use,all,entity,is,port,in,out,end,architecture,of,
    begin,and, AND, or, OR, Not, not, NOT, downto,ALL, PORT, to, process, if, elsif, else, signal, Integer, loop, for, return, function, variable, Component, case, when, others, then, null 
  },
  %alt lyserødt (bordeaux rød?)
  morekeywords=[2]{
    STD_LOGIC_VECTOR,STD_LOGIC,IEEE,STD_LOGIC_1164,
    NUMERIC_STD,STD_LOGIC_ARITH,STD_LOGIC_UNSIGNED,std_logic_vector,
    std_logic, STD_LOGIC_1164, numeric_std,to_unsigned,andOrVectors
  },
  morecomment=[l]--,
  morecomment=[s][\color{codegray}]{"}{"}
}
\lstset{language=VHDL,texcl=true}

\lstdefinestyle{vhdl}{
  language     = VHDL,
  basicstyle   = \footnotesize \ttfamily,
  keywordstyle = [1]\color{keyword}\bfseries,
  keywordstyle = [2]\color{stringPurple}\bfseries,
  commentstyle = \color{comment},
  tabsize=3		                   % sets default tabsize to 2 spaces
}


\definecolor{Python:commentGreen}{RGB}{34,139,24}
\definecolor{Python:stringPurple}{RGB}{208,76,239}
\colorlet{Python:keyword}{blue!100!black!80}
\colorlet{Python:comment}{green!60!black!100}



\lstdefinelanguage{python}{
    %alt blåt
  morekeywords=[1]{
    access,and,break,class,continue,def,del,elif,else,except,exec,finally,for,from,global,if,import,in,is,lambda,not,or,pass,print,raise,return,try,while
  },
  %alt lyserødt (bordeaux rød?)
  morekeywords=[2]{
    abs,all,any,basestring,bin,bool,bytearray,callable,chr,classmethod,cmp,compile,complex,delattr,dict,dir,divmod,enumerate,eval,execfile,file,filter,float,format,frozenset,getattr,globals,hasattr,hash,help,hex,id,input,int,isinstance,issubclass,iter,len,list,locals,long,map,max,memoryview,min,next,object,oct,open,ord,pow,property,range,raw_input,reduce,reload,repr,reversed,round,set,setattr,slice,sorted,staticmethod,str,sum,super,tuple,type,unichr,unicode,vars,xrange,zip,apply,buffer,coerce,intern
  },
  morecomment=[l]\#,
  morecomment=[s][\color{codegray}]{"}{"}
}
\lstset{language=python,texcl=true}


\lstdefinestyle{python}{
  language     = Python,
  basicstyle   = \footnotesize \ttfamily,
  keywordstyle = [1]\color{Python:keyword}\bfseries,
  keywordstyle = [2]\color{Python:stringPurple}\bfseries,
  commentstyle = \color{Python:comment},
  tabsize=3
}




% ¤¤ Kapiteludssende ¤¤ %
\definecolor{numbercolor}{gray}{0.7}		
% Definerer en farve til brug til kapiteludseende
\newif\ifchapternonum

						
% Valg af kapiteludseende - Google 'memoir chapter styles' for alternativer

% ¤¤ Sidehoved/sidefod ¤¤ %
%\makepagestyle{Uni}							

%\nouppercaseheads											
% % Ingen Caps oenskes

		



%%%% EGNE KOMMANDOER %%%%
% ¤¤ Billede hack ¤¤ %									
% Indsaet figurer nemt med \figur{Stoerrelse}{Fil}{Figurtekst}{Label}
\newcommand{\figur}[4]{
		\begin{figure}[H] \centering
			\includegraphics[width=#1\textwidth]{billeder/#2}
			\caption{#3}
			\label{#4}
		\end{figure} 
}

% ¤¤ Specielle tegn ¤¤ %
%\newcommand{\dec}{$^{\circ}$}							
% '\dec' returnerer et gradtegn (husk $$ udenfor aligns)

%\newcommand{\decC}{^{\circ}\text{C}}					
% '\decC' returnerer et gradtegn + 'C' (husk $$ udenfor aligns)

%\newcommand{\m}{\cdot}									
% '\m' returnerer et gangetegn

% Flueben og kryds %
\usepackage{amssymb}% http://ctan.org/pkg/amssymb
\usepackage{pifont}% http://ctan.org/pkg/pifont
\newcommand{\cmark}{\ding{51}}%
\newcommand{\xmark}{\ding{55}}%

%%%% ORDDELING %%%%
\hyphenation{In-te-res-se e-le-ment}

%New colors defined below
\definecolor{codegreen}{rgb}{0,0.6,0}
\definecolor{codegray}{rgb}{0.5,0.5,0.5}
\definecolor{codepurple}{rgb}{0.58,0,0.82}
\definecolor{backcolour}{rgb}{0.95,0.95,0.95}
\definecolor{airforceblue}{rgb}{0.36, 0.54, 0.66}
\definecolor{awesome}{rgb}{1.0, 0.13, 0.32}
\definecolor{azure(colorwheel)}{rgb}{0.0, 0.5, 1.0}
\definecolor{emerald}{rgb}{0.31, 0.78, 0.47}
\definecolor{gray-asparagus}{rgb}{0.27, 0.35, 0.27}
\definecolor{dimgray}{rgb}{0.6, 0.6, 0.6}
\definecolor{tableGray1}{rgb}{0.9, 0.9, 0.9}
\definecolor{tableGray2}{rgb}{0.95, 0.95, 0.95}
\definecolor{dkgreen}{rgb}{0,0.6,0}
\definecolor{gray}{rgb}{0.5,0.5,0.5}
\definecolor{mauve}{rgb}{0.58,0,0.82}


\makeatletter
\providecommand\add@text{}
\newcommand\tagaddtext[1]{%
  \gdef\add@text{#1\gdef\add@text{}}}% 
\renewcommand\tagform@[1]{%
  \maketag@@@{\llap{\add@text\quad}(\ignorespaces#1\unskip\@@italiccorr)}%
}
\makeatother

\usepackage{graphicx,wrapfig,lipsum}

\usepackage{enumitem}

\usepackage{titlesec}

\titleformat{\subsection}
  {\normalfont\normalsize\bfseries\color{Black}}
  {\thesubsection}
  {1em}
  {}
  [{\titlerule[0.8pt]}]

%Definere hvordan listings er sat op
\lstdefinestyle{mystyle}{
  backgroundcolor=\color{backcolour},
  captionpos=b,                    
}
\lstset{style=mystyle}
\renewcommand{\lstlistingname}{Code snippet}