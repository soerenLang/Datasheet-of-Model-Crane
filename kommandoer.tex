\newcommand{\itwoc}{I$^2$C}
% ¤¤ Specielle tegn ¤¤ %
\newcommand{\dec}{^{\circ}}							
% '\dec' returnerer et gradtegn (husk $$ udenfor aligns)

\newcommand{\decC}{^{\circ}\text{C}}					
% '\decC' returnerer et gradtegn + 'C' (husk $$ udenfor aligns)

\newcommand{\m}{\cdot}									
% '\m' returnerer et gangetegn

\newcommand{\coloumn}{\text{C}}
\newcommand{\tesla}{\text{T}}
\newcommand{\newton}{\text{N}}

% \rasppi skriver Raspberry Pi  i rapporten (dovenskab)
\newcommand{\rasppi}{Raspberry Pi }
\newcommand{\amega}{Arduino Mega }

\newcommand{\cpp}{C++}

\newcommand{\euler}{\text{e}}
\newcommand{\ms}{\meter\per\second}


\tikzset{
    right angle quadrant/.code={
        \pgfmathsetmacro\quadranta{{1,1,-1,-1}[#1-1]}     % Arrays for selecting quadrant
        \pgfmathsetmacro\quadrantb{{1,-1,-1,1}[#1-1]}},
    right angle quadrant=1, % Make sure it is set, even if not called explicitly
    right angle length/.code={\def\rightanglelength{#1}},   % Length of symbol
    right angle length=2ex, % Make sure it is set...
    right angle symbol/.style n args={3}{
        insert path={
            let \p0 = ($(#1)!(#3)!(#2)$) in     % Intersection
                let \p1 = ($(\p0)!\quadranta*\rightanglelength!(#3)$), % Point on base line
                \p2 = ($(\p0)!\quadrantb*\rightanglelength!(#2)$) in % Point on perpendicular line
                let \p3 = ($(\p1)+(\p2)-(\p0)$) in  % Corner point of symbol
            (\p1) -- (\p3) -- (\p2)
        }
    }
}


